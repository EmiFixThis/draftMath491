The following table shows a few examples of cryptosystems which can be solved using Shor's algorithm, we refer to them as quantum unsafe, likewise if a problem seems to provide some degree of security against attack by a general purpose quantum computer we refer to it as \emph{quantum safe}.


Any sufficiently large general purpose quantum computer can `easily' solve the mathematical problems below using Shor's or another quantum algorithm. 

\begin{center}
\begin{table}
\begin{tabularx}{300pt}{|X|X|} \hline
        	\textbf{Mathematical Problem} & \textbf{Cryptographic Scheme} \\ \hline
        	Integer Factorization & RSA \\ \hline
        	Discrete Logarithm & Diffie-Hellman Protocol \\ \hline
        	& The Digital Signature Algorithm (DSA) \\ \hline
        	& ElGamal (PGP) \\ \hline
        	Elliptic Curve Discrete Logarithm & Elliptic Curve Diffie-Hellman (ECDH) \\ \hline
       	& Elliptic Curve Digital Signature Algorithm \\ \hline
\end{tabularx}
\caption{Classical Algorithms Susceptible to Quantum Attack}
\label{tb:classalgo}
\end{table}
\end{center} 

\bigskip
    
\begin{center}
\begin{table}
\caption{Efffect of Quantum Computing on Current Cryptosystems}
\begin{tabularx}{300pt}{|X|X|} \hline
\textbf{Survive} & \textbf{Die} \\ \hline
    Merkle hash-tree \\ PKC signature Scheme & DSA \\ \hline
    McEliece hidden-Goppa-code \\ PKC Encryption Scheme & ECDSA \\ \hline
    Lattice-based Cryptography \\ (NTRU) & EDD (n general) \\ \hline
    Multivariate Quadratic Equations \\ (Patarin HFE^{v-} PKC Sig scheme) & HECC (in general) \\ \hline
    Secret Key Crypto \\ Rijndael (AES) & Buchmann-Williams \\ \hline
    & Class groups in general \\ \hline
\end{tabularx}
\label{tb:pq}
\end{table}
\end{center}

\bigskip    
\subsection{Grover's Algorithm}


Grover’s algorithm is a probabilistic quantum search algorithm which is able to find the unique input of a\gls{black box} function $f$, which maps to a unique output in $\mathcal{O}(N^{1/2})$ calculations, 
where $N = |Dom(f)|$. 
It was shown to be optimal by \cite{Ben1997}, with only $\mathcal{O}(N^{1/2})$ minimal number of evaluations required to find a solution. Despite that Grover’s is a probabilistic algorithm the number of evaluations expected before a solution is found is constant and not dependent upon $N$. 

Grover's only speeds up the computation of a black box function only quadratically. For large enough domains, $N$ is fast enough to allow brute force solutions of 128 bit symmetric keys in approximately $2^{64}$ calculations, 256 bit in $2^{128}$. 

Since Grover’s is a search algorithm (usually defined for databases) it is possible to consider the input and output values of $f$ as corresponding database entries and `search’ for one or more $x$ values to satisfy a given $y$ value. That is, we can think of Grover’s as performing an inversion on $y=f(x)$ and returning the correct $x$. 

\begin{center}
\begin{figure}
\begin{align*}
 \Qcircuit @C=1em @R=.7em {
                   &         &                      &                         &                      & \ustick{\text{Grover diffusion operator}} \\
  \lstick{\ket{0}} & /^n \qw & \gate{H^{\otimes n}} & \multigate{1}{U_\omega} & \gate{H^{\otimes n}} & \gate{2 \ket{0^n}\bra{0^n} - I_n}         & \gate{H^{\otimes n}} & \qw & \cdots & & \meter & \cw \\
  \lstick{\ket{1}} & \qw     & \gate{H}             & \ghost{U_\omega}        & \qw                  & \qw                                       & \qw                  & \qw & \cdots & \\
                   &         &                      &                         &                      & \dstick{\text{Repeat $O(\sqrt{N})$ times}}
  \gategroup{2}{5}{2}{7}{.7em}{^\}}
  \gategroup{2}{4}{3}{10}{.7em}{_\}}
 }
\end{align*}
\end{figure}
\caption{Quantum Circuit Representation of Grover's Algorithm}
\label{dia:grovers}
\end{center}
\cite{Ben2011}

	