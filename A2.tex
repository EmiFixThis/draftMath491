\section{Appendix B: Base Security Objectives}

There are four basic security objectives that must be considered when constructing any system concerned with securing data or information. These four basic definitions allow for the derivation of all other security objectives which may or may not be necessary to ensure the security of information for a given system.

\begin{itemize}
    \item \textbf{Confidentiality:} The objective of confidentiality ensures that unauthorized users will not be purposefully (or accidentally) give access to resources protected by the system. 
    \item \textbf{Integrity:} Ensures that the resources are preserved, used, and appropriately maintained throughout their life-cycle under the system. That is, any data is not alterable in an undetectable manner, retains the same
accuracy as its created date (or registered modification date), and is complete with respect to its creation and activity log. 
    \item \textbf{Availability:} Ensuring resources are available to authorized parties as required. 
    \item \textbf{Non-repudiation:} Prevention of commitment or action denial.
\end{itemize}

\subsection{Derived Security Objectives}

Each implementation of a scheme requires a flexible set of security objectives which are dependent upon the context the system and it's users will employ \textbf{TODO:change wording}. The base definitions for the objectives allow the derivation of all other security objectives.

\begin{table}[]
\centering
\caption{Security Objectives}
\label{secObjTable}
\begin{tabular}{|l|l|}
\hline
\bf{Confidentiality}        & Keeping information in the hands of only authorized or trusted users.                                                      
\\ \hline
\bf{Data Integrity}         & Ensuring information remains unaltered by unauthorized parties                                                             \\ \hline
\bf{Identification}         & Corroborating the identity of a user, system, account, etc.                                                                
\\ \hline
\bf{Message Authentication} & Corroborating the source of information.                                                                                   \\ \hline
\bf{Signature}              & An object which assigns information to a user.                                                                             \\ \hline
\bf{Authorization}          & Permissions given to another user to allow them to use or be something.                                                    
\\ \hline
\bf{Validation}             & An object which ensures that authorization is provided in an appropriate time frame.                                       
\\ \hline
\bf{Access Control}         & Restricting the access of resources to only authorized users.                                                              
\\ \hline
\bf{Certification}          & Information endorsed by an authorized party.                                                                               \\ \hline
\bf{Timestamping}           & A record of the creation. existence, or modification of information.                                                       \\ \hline
\bf{Witnessing}             & A verification of the creation, existence, or modification of information by a user who did not take part in the activity. 
\\ \hline
\bf{Receipt}                & Confirmation that information has been received.                                                                           \\ \hline
\bf{Confirmation}           & Affirmation of a provided service.                                                                                         \\ \hline
\bf{Ownership}              & Provision of usage or transfer rights to a user.                                                                           \\ \hline
\bf{Anonymity}              & A process which conceals the identity of a user.                                                                           \\ \hline
\bf{Non-repudiation}        & Prevention of commitment or action denial.                                                                                 \\ \hline
\bf{Revocation}             & Allows the privileges of an authorized user to be reversed.                                                                                                                            \\ \hline
\end{tabular}
\end{table}
