\section{Appendix C: Definitions for Quantifying Hardness in Cryptography}


\begin{defn}{Hardness}
Let $\mathcal{P}$ be a problem such that $\mathcal{P}$ is hard, and let $\mathcal{S}$ be a cryptographic scheme. 
We say that $\mathcal{S}$ is secure, if there does not exist a Probabilistic Polynomial Time (PPT) algorithm $\mathcal{A}$, which solves $\mathcal{P}$. 
\end{defn}	


\begin{defn}{Tightness}
The measure of the quality of a particular security reduction is called its tightness.
If a security reduction is tight, then the scheme is \emph{at least as hard} as the mathematical problem on which it is based. 
If the reduction is not tight, then the scheme is \emph{possibly easier} than the problem which it is based on. 
\end{defn}
	

\begin{defn}{Bit Hardness}
A measure of the hardness of a mathematical problem to be used as a basis for some cryptographic scheme is called it’s bit hardness. 
\end{defn}

