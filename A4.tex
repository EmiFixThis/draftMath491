\section{Appendix D: Time Complexity}

	Time complexity quantifies the amount of time an algorithm requires to solve a problem as a function of the length of the input it is provided. 
    
    The time complexity is typically measured by counting the number of operations performed, where the amount of time taken by each operation is fixed, and the number of operations is known. 
    The time complexity can then be represented asymptotically by allowing the input $n$ to approach infinity.

However, algorithms may not be well behaved for all types of input, even when they are the same size. That is, the performance (runtime), $T$ of an algorithm $A$, may vary greatly for different types or sizes of inputs $n$. An algorithm is  classified according to its runtime behavior for an arbitrary input size. 

These behaviors are typically given in terms of the \emph{worst-case} time complexity of an algorithm, $T(n)$. 

\begin{defn}{Worst-case Time Complexity}
	
    The worst-case time complexity of an algorithm is the largest possible amount of time a given algorithm may require in order to solve a problem. 
\end{defn}


