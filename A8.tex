\section{Appendix G: Basic Definitions for Rings}


A ring $\mathcal{R}$ is a group defined for two operations, instead of only one operator. A ring has both additive operation, and a multiplicative operations. 

Denoted:

$\mathcal{R}(+,\ast)$. 


Let $\mathbb{F}$ be a field, then $\mathbb{F}$ is a ring where the nonzero elements of $\mathbb{F}$ form an abelian group under multiplication. We can do anything in this field we can in the reals, complex, or rational numbers we are familiar with; we can add, multiply, subtract, and divide. In fact the set of reals, complex, and rational numbers are all fields themselves.


A multiplicative group is is the set of all invertible elements of a field, also a ring, or some other structure having a multiplicative operation on it's elements. For example, $\mathbb{Z}/n\mathbb{Z}$ is the multiplicative group of integers mod $n$; i.e., the elements of $\mathbb{Z}/n\mathbb{Z}$ which are invertible under multiplication. 


\begin{nt}
    
    Fields $\subset$ groups,
    Rings $\subset$ groups, 
    Fields $\subset$ rings, 
   	
    Notice: 
    \textrm{While all rings are groups (abelian groups under addition), the converse is not always true, all rings are not  fields.}
\end{nt}


    
A ring is a set $\mathcal{R}$ defined on two operations $+$ and $\ast$, which satisfy the following properties:
    
\begin{itemize}
    \item The additive identity $0$ exists,
    \item the elements of $\mathcal{R}$ commute over addition,
    \item they also are associative over addition,
    \item an additive inverse exists for every element in $\mathcal{R}$.
\end{itemize}

These four properties define an abelian group.
The next three properties are defined similarly, with an exception of the multiplicative inverse.


\begin{itemize}
    \item A multiplicative identity exists, $1$.
    \item the elements of $\mathcal{R}$ commute over multiplication,
    \item they also are associative over multiplication.
    \item The distributive law, holds. (It's the glue that connects these two operations together).             
\end{itemize}
\footnote{The distributive property: the `jelly side down' of ring theory.}





