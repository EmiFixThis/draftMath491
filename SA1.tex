\section{Introduction}

\todo[inline]{I'm not sure "asymmetric-key/symmetric-key" need to be hyphenated?}
Modern cryptographic schemes can generally be divided into two classes: asymmetric-key schemes and symmetric-key schemes. 

Symmetric-key schemes involve only one secret key between two communicating parties called a \emph{shared secret}. In contrast, asymmetric-key schemes contain two secret keys, one for each party, and two public keys which are publicly distributed.
Symmetric schemes are in general more efficient than asymetric schemes but have the disadvantage that keys must be agreed upon by parties who wish to communicate in advance. Specifically symmetric-key schemes require that secret keys be distributed over secure channels \footnote{In real world applications there is no such thing as a secure channel.}\todo[inline]{I might remove this footnote - there are secure channels (e.g. whispering in your ear).}. These types of key distribution schemes are not compatible with most modern methods of communication over networks, which are by and large conducted over the internet, an insecure channel.


In contrast, Public-key cryptography is an asymmetric-key class of systems which allows parties which have never met, much less agreed upon secret keys in advance, to communicate privately over insecure channels\todo{Rework the first sentence of this paragraph; it reads a bit awkwardly}. The primary concern is to make communication by public key cryptographic schemes as efficient, secure, and easy to implement as those of symmetric key systems \footnote{Notably, the problem of determining if any message has indeed been communicated securely (or if it has not) is an open problem for which no known solutions exist that cannot be classified as temporary hacks.}.


In this context, implementation of a particular cryptographic scheme is often a two-fold problem. That is, on one hand the cryptosystem must be constructed with much thought in mind to its efficiency and economy, on the other hand improper assumptions concerning the security imparted by the system must always be given precedence as any failure in this regard invalidates the inherent purpose of the system's existence, making all other efforts with respect to the system trivial in it's wake. Since the purpose of cryptography is to keep communications and information private the role in managing the risks associated with the system must be weighted necessarily heavier towards ensuring the fulfillment of a given set of security objectives. A balance between efficiency and security must be carefully, and continuously, maintained throughout construction, implementation, and daily maintenance of the system. To even momentarily neglect this precarious balancing act courts disaster. 

\todo[inline]{Do you think there needs to be stuff about classial/quantum/post-quantum crypto in the intro? Or at least a bit on lattice crypto? The TOC looks like the paper is mostly devoted to that, but the intro doesn't mention it.}
