\section{Counterexamples Win Out}


This story begins at one such occurance of forgetfulness in modern cryptography. Where the balancing act between efficiency and security in lattice-based cryptography was for a time forgotten, and constructions by many were weighted towards efficiency. Particularly in ideal lattice-based cryptography, where a great many researchers took for granted the level of security lattice-based schemes were conjectured to impart, focusing their attention on the efficiency of their schemes instead. These researchers attempted construction of more and more efficient schemes, assuming that the possiblity their systems were as secure as schemes defined over more general lattices, would eventually be proven,... by someone else. 


In mathematics the (and possibly life in general), the best possible refutation of a claim something does not or cannot exist is a counterexample. Such an example stands constructed in all its defining premises, defiantly and conclusively proving \emph{"I exist".} Cryptography is not immune from these somewhat ammusing and ironic examples, and lattice cryptography is no exception to this rule, where these counterexamples often show themselves in the form of use cases.


The faith of the ideal lattice cryptographers in the strength of their efficient constructions is not a new mistake in the history of cryptography. This assumption of strength is famously mirrored in the arogance of engineers of the German Enigma ciphers of World War II, stubbornly clinging to the calculations of theoretical strength despite all evidence to the contrary. Doubtless, history professors with interest in such affairs would be fully justified in gloating and repetition of the adage \emph{"those who forget history are doomed to repeat it,"} with all the smuggness they can bear. 


There is no better such example in modern lattice-based cryptography then the Soliloquy problem; an unexpected announcement by British researchers in the Communications-Electronics Security Group (CESG) (the information assurance division of the Government Communications Headquarters (GCHQ)). This informal report dubbed \textit{Soliloquy: A Cautionary Tale} [\cite{Cam20140}], brought on a slew of uncharacteristically emotional responses in the cryptographic community, a mixture of shock, speculation, as well as a few cases of well-earned, smug \emph{"I told you so"} sentiments. As for researchers working towards efficient constructions of similar schemes there was a backlash of reactions ranging the spectrum of in one direction, unsurprised acceptance and in the opposite direction, rage and denial. 

The announcement by CESG, vaguely outlined a cryptosystem given over ideal lattices of characteristic two, a characteristic much overused despite the warnings of more experienced researchers [\cite{Cra20151}] that efforts in other fields were necessary in order to guard against the possibility that the particular field characteristic was shown to exhibit such a vulnerability. 



