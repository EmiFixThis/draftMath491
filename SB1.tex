\section{Introductory Cryptographic Terminology and Notation}


Public Key Cryptography (PKC) solves the problem of how to enable two people who have never met to communicate securely over insecure channels. 
 

Public-key depends on the following assumptions: 
\begin{enumerate}
 \item In the forward direction, we assume there exists a one-way function which is easy to compute.
 \item In the reverse direction, we assume that the same function is hard to invert. 
 \end{enumerate}
 
 
 
\todo{INSERT PKC DIAGRAM}

\subsection{Asymmetric \& Symmetric Cryptography} 

The concepts of symmetric and asymmetric cryptographic-key schemes are given below as independent forms. However, it is rarely the case that these schemes will be applied in this manner particularly in a Public-key asymmetric system. Both asymmetric and symmetric key schemes are assumed to be classical cryptosystems. By way of classical we mean the systems are defined in terms of the chances the mathematical problems these systems are based upon have of being solved by an adversarial system. I.e., The liklihood of breaking the security of the ciphers in any of these systems is dependent on the assumed capabilities of the adversary's system. The common assumptions attributed to an adversary are as follows:

\begin{itemize}
    \item The adversary is in possession of (or has access to) slightly more than a reasonable (but still finite and non-quantum) amount of computational resources;
    \item If the adversarial system (as described above) provides a solution to the problem serving as a basis for security of the target system, it does so by applying a probabilistic, polynomial-time, algorithm. 
\end{itemize}

\subsubsection*{Symmetric-key Cryptography} 

In cannonical symmetric-key schemes only one key is generated for both the encryption and decryption algorithms. Symmetric-keys must therefore be exchanged over secure channels where both communicating parties have agreed in advance to the method of encryption and have exchanged secret keys. 


Formally, we say the two parties are in possesion of a \textit{shared secret} and each party is mutually, as well as equally responsible for the maintenance of the secret which establishes and secures their communications. 

\todo{Is this supposed to be here, I think this sentence is supposed to be in asym.} In certain contexts this method can be made to be as secure as an asymmetric key system. It's main drawback is the shared secret which has the ability to invalidate the integrity of communications if either party is compromised.


Additionally, since the method requires a secure channel as well as a shared secret both parties must know they will have a need for encrypted communication in advance and negotiate the means of transmission and key exchange prior to communicating. This means the parties must know each other prior to communicating. 



\subsubsection*{Formal Definitions for Symmetric-key Schemes}


We define a symmetric-key scheme as follows:


Let $\mathbf{\mathcal{K}}$ be the set of all possible keys $k$  generated by a key generation function, called the \textit{keyspace}.

The key generation algorithm $\mathbf{Gen}$ is a probabilistic function which uses a distribution chosen which is appropriate with respect to the scheme.

The key generation algorithm randomly chooses a key $k$ from it's distribution uniformly. 


Let $\mathbf{\mathcal{M}}$ be the set of all possible messages $m$ called the \textit{message} or \textit{{plaintext space}}.


The encryption function $\mathbf{{Enc}}$ takes a key $k$ and message $m$ as input producing a ciphertext $c$.


The set of all possible ciphertexts (The ciphertext space) is denoted $\mathbf{\mathcal{C}}$, defined by taking the pair of spaces $\mathbf{(\mathcal{K}, \mathcal{M})}$ together. 


Let $\mathbf{Dec}$ be the decryption function which takes a ciphertext $c$ and key $k$ as input returning the corresponding message $m$.


\subsubsection*{Cryptosystem}


Taking the above components together we have the following defintion of a symmetric-key cryptosystem using the traditionally chosen communicating parties \emph{Alice, Bob} and \emph{Eve (the eavesdropper)}:


A symmetric-key cryptosystem is given by a set of algorthms $\mathbf{(Gen, Enc, Dec)}$ and the message space $\mathcal{M}$.

The cryptosystem performs it's functions in the following order:

First, Alice and Bob use $Gen$ to produce a shared secret key $k$.

Whenever Bob wishes to send a message $m$ to Alice (or Alice to Bob) he inputs $(m,k)$ into $Enc$ to get a ciphertext of the message $c$.
\[c \coloneqq Enc_{k}(m)\]

When Alice recives the ciphertext $c$ from Bob she inputs both $(c,k)$ into the decryption function and gets back the plaintext message $m$.
\[Dec_{k}(c) \coloneqq m.\]

To ensure the cryptosystem is correct, that is that it holds \textit{for every} message key pair $(k,m) : k \in \mathcal{K}$ and $m \in \mathcal{M}$ you must show that using the key on a ciphertext outputs the correct message. 

\[Dec_{k}(Enc_{k}(m)) = m\]

\[Dec_{k}(c) = m : Enc_{k}(m) = c\]



\subsubsection*{Asymmetric-key Cryptography}
In contrast symmetric-key schemes are determined by two sets of keys one for each party. Each set consists of a private key known only to the owner and a \emph{public key} which can be widely distributed.
Public-key uses a \emph{one-way function} as its public key; this function is easy to compute but hard to invert and is available publicly for use as an encryption method for the key owner. A second key called the private key is known only to the key owner; this key is a \emph{trapdoor function} which can invert a ciphertext back to its plaintext form.

\todo{Insert equations as above}
 
\subsection{Connecting Symmetric \& Asymmetric Cryptography}
It seems at first glance that asymmetric and symmetric cryptosystems are isolated methods of encryption. However, symmetric encryption is well-suited for the task of assisting asymmetric key schemes not only for efficiency purposes but also in the role of key management. 
