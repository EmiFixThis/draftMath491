\section{A Brief History of Lattice-based Cryptography}

\todo{Rework this section altogether}

\subsection*{Lattices (Informally)}
\todo{This subsection gives a brief history of lattices before their modern uses.}

Before lattices became popular in cryptography circles they were in terms of mathematical history by and large mocked, ridiculed, maybe even hated. Lattices can be traced back to Gauss' circle problem (a favorite IMO strategy for solving problems in synthetic geometry), which asks \textit{given a circle $\mathcal{R}^{2}$ centered at $\mathcal{O}$ how many integer lattice points $(m,n)$ are inside $\mathcal{R}$?}


In 2009, lattice cryptography began a swift resurgance in popularity after Oded Regev's new problem defined over lattices called \emph{Learning with Errors} (LWE) [\bib{Reg2009}] proved to be as hard to solve as certain worst-case lattice problems. Learning with Errors is a generalization of the \emph{Parity Learning with Noise} (PLN) problem which is conjectured to be hard to solve. \footnote{Both PLN and LWE are described in section \todo{Insert Section Number Here}}.
 
 

The introduction of lattices in cryptography brought in new, enthusiastic researchers and propelled experienced researchers to the forefront whose advice, often repeated, was misunderstood and in many instances ignored [\bib{Cra20151}].


\sidenote{It is worth explicitly stating that cryptography is inextricably bound and defined by its relationship with security objectives and privacy. These constraints not only benefit from the diversification of schemes from their original construction, they demand it.}
 for diversification was so often ignored by researchers in favor of
achieving efficiency it became increasingly difficult to find
constructions which had not been constructed over a particular number
field; more over research had focused not simply to this number field
it had narrowed to the point of most constructions being defined
specifically over the cyclotomic field of characteristic two [\bib{Cra20151}].
It is perhaps lucky that the discovery by the researchers at CESG was
not only disclosed publicly (the agency does not in general reveal much
less publish their findings), but that the defect was not found as a
consequence of an exploit, or even more at some point in the future
when cryptosystems having this type of construction may be widely
implemented and assumed secure.

Traditionally cryptography has been a two player game between
cryptographers and cryptanalysts. However, cryptanalysis has
diversified over time leaving cryptographers to play defense against an
increasingly diverse set of cryptanalysts on the offense.


