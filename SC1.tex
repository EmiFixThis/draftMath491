\section{Semantic Security: Defining Security with Games} 

\subsection{Goldwasser \& Micali's: Mental Poker and Partial Information} 


Semantic Security was a game proposed by Goldwasser and Micali in their 1982 paper entitled: \textit{"Probabilistic Encryption \& How to Play Mental Poker Keeping Secret All Partial Information"}[\cite{Gol19820}], in response to papers from Rivest, Shamir, Adelman, [\cite{Riv19780}] and Rabin [\cite{Rab19790}] detailing the theory behind the Diffie-Hellman based RSA cryptosystem. 

Goldwasser and Micali's intention was to strengthen the security assumptions of Diffie-Hellman based cryptosystems by formally defining what it would mean for such systems to be called secure. 


[\cite{Gol19820}] Proposed the following property for any implementation of a Diffie-Hellman Public-key Cryptosystem:  
\begin{prop}{"An adversary, who knows the encryption of an algorithm and is given the cyphertext, cannot obtain any information about the cleartext."} \footnote{\cite{Gol1982} Pg.1, Paragraph 1, Lines 3-6} 
\end{prop}


Informally, Goldwasser and Micali's property says that a given cryptographic scheme is considered insecure if given some of the ciphertext (but not the private key), it is possible for an adversary to recover any information about the plaintext, or any recover useful information about the plaintext of the message, by some manipulation of the ciphertext over a reasonable amount of time. 


\subsection{Weaknesses in the Assumptions of RSA} 

\todo[inline]{I'm not sure a section about weaknesses of RSA belongs here? It seems lke you should focus on defining/explaining semantic security?}

\todo[inline]{You need to somehow introduce or define RSA, even if it's just a single sentence. You probably shouldn't just jump in to talking about weaknesses in RSA when you haven't mentioned it yet before now.}
Goldwasser and Micali pointed out that the security assumptions given in [\cite{Rab19790}] and  [\cite{Riv19780}] had some particularly significant weaknesses that shouldn't be overlooked.


\begin{asu}
There exists a trapdoor function $f(x)$ that is easily computed, while $x$ is not easily computed from $f(x)$ unless some additional information is known. To encrypt any message $m$, you evaluate $f$ at $m$, and receive the ciphertext $f(m)$.
\end{asu}

[\cite{Gol19820}] give two weaknesses of this first assumption:


\begin{rem}
 If $x$ has some specialized form, the fact that $f(x)$ is assumed to be a trapdoor function does not guarantee that $x$ cannot be computed. There are two important points in this weakness:
\end{rem}

\begin{case}
First, messages do not typically consists of randomly chosen numbers, they have more structure.
\end{case}

\begin{case}
Second, the additional structure these messages posess may assist in decoding the ciphertext.
\end{case} 

\begin{exmp}
Compare two functions $f(m)$ and $g(n)$, where the input $m$ for $f$ is some random ASCII sequence, and the input $n$ for $g$ is an ASCII sequence which represents sentences written in English. Then it may happen that $m$ is hard to recover from $f(m)$, but that $n$ is easy to recover from $g(n)$.\todo[inline]{This example maybe needs to be elaborated a bit more?}
\end{exmp}

\begin{rem}
Even if we assume that $f$ is a trapdoor function, that does not negate the possibility that the message or even half the message may not be recoverable from $f(x)$. 
\end{rem}


[\cite{Gol19820}] state the following concerning the use of trapdoors in general:
\begin{quotation}
    Covering ones face with a handkerchief certainly helps to hide personal identity. However:
    \begin{enumerate}
         \item It will not hide from me the identity of a special subset of people: my mother, my sister, close freinds.
         \item I can gather a lot of information about people I cannot identify: their height, their hair color, and so on.
    \end{enumerate}
\end{quotation}

These facts about the use of a handkerchief are exactly those you would find issue with in using any Public-key cryptosystem whose security relies on trapdoors. In fact, they are the same arguments which arise as problems in cases one and two. 

Goldwasser and Micali suggest the following changes to the RSA and Rabin schemes:


First, they declare the set of messages $\mathcal{M}$ should be sparse.

\begin{cla}
    If the set of messages $\mathcal{M}$ is sparse in $\mathbb{Z}_{N}^{*}$, then the ability to correctly recover even 1\% of all messages is not possible for a random polynomial time algorithm in the case of factoring.
\end{cla}

Where \emph{sparse} in this context means that when $x \in \mathbb{Z}_{N}^{*}$ is chosen randomly, we have a probability of any $x$ chosen being a message is for all intents and purposes zero. 


Second, they define the notion of secure for any Public-key cryptosystem in a way that disallows the use of trapdoor functions:

\begin{defn}
    Let $\mathcal{P}$ 
\end{defn}
