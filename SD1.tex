\section{A Brief History of Lattice-based Cryptography}

\todo{Rework this section altogether}

\subsection*{Lattices (Informally)}
\todo{This subsection gives a brief history of lattices before their modern uses.}

Before lattices became popular in cryptography circles they were in terms of mathematical history by and large mocked, ridiculed, maybe even hated. Lattices can be traced back to Gauss' circle problem (a favorite IMO strategy for solving problems in synthetic geometry), which asks \textit{given a circle $\mathcal{R}^{2}$ centered at $\mathcal{O}$ how many integer lattice points $(m,n)$ are inside $\mathcal{R}$?}


In 2009, lattice cryptography began a swift resurgance in popularity after Oded Regev's new problem defined over lattices called \emph{Learning with Errors} (LWE) [\cite{Reg2009}] proved to be as hard to solve as certain worst-case lattice problems. Learning with Errors is a generalization of the \emph{Parity Learning with Noise} (PLN) problem which is conjectured to be hard to solve. \footnote{Both PLN and LWE are described in section \todo{Insert Section Number Here}}.
 
 

