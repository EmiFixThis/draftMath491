\subsection{Hardness Assumptions}

We define the three canonical types of complexity problems: \emph{best-case, average-case}, and \emph{worst-case} known as \emph{hardness assumptions}. 

\begin{defn}
    \textbf{Best-case:} 
    Refers to the set of problems $\mathcal{P}$ which (uninterestingly enough) have perfect conditions, and outcomes.
\end{defn}


We can interpret the average-case in common mathematical terms as the quantifier \emph{for some}.

\begin{defn}
    \textbf{Average-case:} 
For some negligible set of random instances, the problem $\mathcal{P}$ may not be hard to solve. 
\end{defn}


The worst-case can be defined as the contrapositive of the average-case, which changes the interpretation in a similar way, as the quantifier \emph{for any}, or \emph{for all}, namely:

\begin{defn}
A problem $\mathcal{P}$ is hard to solve (\emph{intractible})\footnote{Tractible: (first known usage 1502) meaning \textit{easily managed}. In contrast, intractible (earliest usage 1545) means difficult to manage. In mathematics, refers to a problem which easily lends itself to solution. Computer Science: A problem which can be solved algorithmically, usually in polynomial time. \todo[inline]{How do you cite Wikitionary?}} in the worst-case, for any positve set of random instances.
\end{defn}