most efficient and advanced systems in both categories rely on the ring-LWE problem \[LPR10], an analogue of the standard learning with errors problem \[Reg05]

Keywords: 
Source: 
Scents: 
Slip 20

In all applications of ring-LWE, and particularly those related to homomorphic encryption, a main technical challenge is to control the sizes of the noise terms when manipulating ring-LWE samples under addition, multiplication, and other operations.

Keywords: 
Source: 
Scents: 
Slip 21

For correct decryption, q must be chosen large enough so that the final accumulated error terms do not “wrap around” modulo q and cause decryption error.

Keywords: 
Source: 
Scents: 
Slip 22

On the other hand, the error rate (roughly, the ratio of the noise magnitude to the modulus q) of the original published ring-LWE samples and the dimension n trade off to determine the theoretical and concrete hardness of the ring-LWE problem.

Keywords: 
Source: 
Scents: 
Slip 23

Tighter control of the noise growth therefore allows for a larger initial error rate which permits a smaller modulus q and dimension n, which leads to smaller keys and ciphertexts, and faster operations for a given level of security.

Keywords: 
Source: 
Scents: 
Slip 24

the search/decision equivalence for ring-LWE in arbitrary cyclotomics \[LPR10] relies on their special algebraic properties, as do many recent works that aim for more efficient fully homomorphic encryption schemes (e.g., \[SV11, BGV12, GHS12a, GHS12b, GHPS12]).

Keywords: 
Source: 
Scents: 
Slip 25

In particular, power-of-two cyclotomics, i.e., where the index m = 2  k for some k ≥ 1, are especially nice to work with, because (among other reasons) n = m/2 is also a power of two, Φ m (X) = X n + 1 is maximally sparse, and polynomial arithmetic modulo Φ m (X) can be performed very efficiently using just a slight tweak of the classical n-dimensional FFT (see, e.g., \[LMPR08]).

Keywords: 
Source: 
Scents: 
Slip 26

`Indeed, power-of-two cyclotomics have become the dominant and preferred class of rings in almost all recent ring-based cryptographic schemes (e.g., \[LMPR08, LM08, Lyu09, Gen09b, Gen10, LPR10, SS11, BV11b, BGV12, GHS12a, GHS12b, Lyu12, BPR12, MP12, GLP12, GHPS12]), often to the exclusion of all other rings.' \cite{Cra2015}

Keywords: 
Source: 
Scents: 
Slip 27

While power-of-two cyclotomic rings are very convenient to use, there are several reasons why it is essential to consider other cyclotomics as well

most obvious, practical reason is that powers of two are sparsely distributed, and the desired concrete security level for an application may call for a ring dimension much smaller than the next-largest power of two.

restricting to powers of two could lead to key sizes and runtimes that are at least twice as large as necessary.

A more fundamental reason is that certain applications, such as the above-mentioned works that aim for more efficient (fully) homomorphic encryption, require the use of non-power-of-two cyclotomic rings.

This is because power-of-two cyclotomics lack the requisite algebraic properties needed to implement features like SIMD operations on “packed” ciphertexts, or plaintext spaces isomorphic to finite fields of characteristic two (other than F 2 itself)."




A final important reason is diversification of security assumptions."



Keywords: 
Source: 
Scents: 
Slip 29

So while we might conjecture that ring-LWE and ideal lattice problems are hard in every cyclotomic ring (of sufficiently high dimension), some rings might turn out to be significantly easier than others."

Unfortunately, working in non-power-of-two cyclotomics is rather delicate, and the current state of affairs is unsatisfactory in several ways.

"Unlike the special case where m is a power of two, in general the cyclotomic polynomial Φ m (X) can be quite “irregular” and dense, with large coefficients."

While in principle, polynomial arithmetic modulo Φ m (X) can still be done in O(n log n) scalar operations (on high-precision complex numbers), the generic algorithms for achieving this are rather complex and hard to implement, with large constants hidden by the O(·) notation."

If one views Z\[X]/(Φ m (X)) as the set of polynomial residues of the form a 0 + a 1 X + · · · + a n−1 X n−1 , and uses the naïve “coefficient embedding” that views them as vectors (a 0 , a 1 , . . . , a n−1 ) ∈ Z  n to define geometric quantities like the \` 2 norm, then both the concrete and theoretical security of cryptographic schemes depend heavily on the form of Φ m (X).
